\documentclass[12pt,a4paper,sans]{moderncv}        % possible options include font size ('10pt', '11pt' and '12pt'), paper size ('a4paper', 'letterpaper', 'a5paper', 'legalpaper', 'executivepaper' and 'landscape') and font family ('sans' and 'roman')

\usepackage{verbatim}
\usepackage{polski}
\usepackage[utf8]{inputenc}
\usepackage[firstyear=2010,lastyear=2020]{moderntimeline}
\usepackage{graphicx}
\usepackage{epstopdf} %%package to overcome problem with eps in pdf files
\usepackage{float}
\usepackage{tikz}
\usepackage[bottom=0.5in]{geometry}
\restylefloat{figure}

\newcommand*{\cvtripleitem}[7][.25em]{%
 \cvitem[#1]{#2}{%
   \begin{minipage}[t]{\tripleitemmaincolumnwidth}#3\end{minipage}%
   \hfill
   \begin{minipage}[t]{\hintscolumnwidth}\raggedleft\hintstyle{#4}\end{minipage}%
   \hspace*{\separatorcolumnwidth}%
   \begin{minipage}[t]{\tripleitemmaincolumnwidth}#5\end{minipage}%
   \hfill
   \begin{minipage}[t]{\hintscolumnwidth}\raggedleft\hintstyle{#6}\end{minipage}%
   \hspace*{\separatorcolumnwidth}%
   \begin{minipage}[t]{\tripleitemmaincolumnwidth}#7\end{minipage}%
   }%
}

\newcommand{\cvtripleitemlogo}[7][.25em]{%
 \cvtripleitem[.25em]
  {#2}
  {\raisebox{-0.3\height}{\includegraphics[width=8mm]{#3}}}
  {#4}
  {\raisebox{-0.3\height}{\includegraphics[width=8mm]{#5}}}
  {#6}
  {\raisebox{-0.3\height}{\includegraphics[width=8mm]{#7}}}
}

\newcommand\githublogo{\raisebox{-1pt}{\includegraphics[height=9pt]{github}}}

% moderncv themes
\moderncvstyle{classic}                             % style options are 'casual' (default), 'classic', 'oldstyle' and 'banking'
\moderncvcolor{blue}                               % color options 'blue' (default), 'orange', 'green', 'red', 'purple', 'grey' and 'black'
%\renewcommand{\familydefault}{\sfdefault}         % to set the default font; use '\sfdefault' for the default sans serif font, '\rmdefault' for the default roman one, or any tex font name
%\nopagenumbers{}                                  % uncomment to suppress automatic page numbering for CVs longer than one page

% adjust the page margins
\usepackage[scale=0.75]{geometry}
%\setlength{\hintscolumnwidth}{3cm}                % if you want to change the width of the column with the dates
%\setlength{\makecvtitlenamewidth}{10cm}           % for the 'classic' style, if you want to force the width allocated to your name and avoid line breaks. be careful though, the length is normally calculated to avoid any overlap with your personal info; use this at your own typographical risks...

% personal data
\firstname{Filip}
\familyname{Czaplicki}
\title{Resumé}                               % optional, remove / comment the line if not wanted
\address{}{}{Warsaw, Poland}% optional, remove / comment the line if not wanted; the "postcode city" and and "country" arguments can be omitted or provided empty
\email{filipczaplicki@gmail.com}                               % optional, remove / comment the line if not wanted
% \homepage{starsep.com}                         % optional, remove / comment the line if not wanted
\extrainfo{\githublogo~\httplink{github.com/starsep}}              % optional, remove / comment the line if not wanted
% \photo[64pt][0pt]{picture}                       % optional, remove / comment the line if not wanted; '64pt' is the height the picture must be resized to, 0.4pt is the thickness of the frame around it (put it to 0pt for no frame) and 'picture' is the name of the picture file
%\quote{Some quote}                                 % optional, remove / comment the line if not wanted

\begin{document}

\makecvtitle

\section{\textsc{Work Experience}}


\textbf{\cventry{2020--present}{Senior Mobile Developer}{}{\href{https://radgost.com/}{\underline{Radgost}}/\href{https://invoiceocean.com/}{\underline{InvoiceOcean}}}{Warsaw, Poland}{}}
\subsection{\textsc{Fakturownia/InvoiceOcean}}
\cvitem{app}{Android Native Fakturownia/InvoiceOcean}
\cvitem{description}{New native mobile app for e-invoicing}
\cvitem{technologies}{
  \href{https://kotlinlang.org/}{\underline{Kotlin}},
  \href{https://developer.android.com/guide}{\underline{Android SDK}},
  \href{https://insert-koin.io/}{\underline{Koin}},
  \href{https://developer.android.com/jetpack}{\underline{Jetpack}},
  \href{https://ktor.io/clients/index.html}{\underline{Ktor}},
  \href{https://cashapp.github.io/sqldelight/}{\underline{SQLDelight}}
  \newline
}

\cvitem{app}{Fakturownia/InvoiceOcean}
\cvitem{description}{Mobile app for e-invoicing}
\cvitem{technologies}{React Native, iOS, Android}
\cvitem{links}{
  \href{https://play.google.com/store/apps/details?id=pl.fakturownia}{\underline{Google Play}},
  \href{https://apps.apple.com/pl/app/invoiceocean/id1114785400}{\underline{AppStore}}
  \newline
}

\textbf{\cventry{2017--2020}{Software Engineer}{}{\href{https://qed.ai/}{\underline{QED}}}{Warsaw, Poland}{}}

\subsection{\textsc{ScanForm}}

\cvitem{app}{ScanForm Web}
\cvitem{description}{Web app for realignment and OCR of scanned forms}
\cvitem{technologies}{Django, Python, Numpy, Pillow, OpenCV, PyTorch}
\cvitem{link}{\url{https://qed.ai/scanform/} \newline}

\cvitem{app}{ScanForm Mobile}
\cvitem{description}{Android App for taking pictures and uploading them to webserver}
\cvitem{technologies}{Kotlin, Android SDK, OpenCV}
\cvitem{link}{\href{https://play.google.com/store/apps/details?id=qed.scanform.scanner}{\underline{Google Play}}\newline}

\newpage

\subsection{\textsc{AutoDrone/Hive}}

\cvitem{app}{AutoDrone}
\cvitem{description}{Android app for automatic drone flights with DJI Phantom}
\cvitem{technologies}{Android, Java, DJI SDK}
\cvitem{links}{\href{https://play.google.com/store/apps/details?id=ai.qed.auto.drone}{\underline{Google
Play}}, \url{https://qed.ai/drones/} \newline}

\cvitem{app}{Hive}
\cvitem{description}{Web app for displaying drone orthomosaics on map \newline
  Orthomosaics are stiched from images taken with AutoDrone}
\cvitem{technologies}{Django, Python, OpenDroneMap}
\newline

\textbf{\cventry{2016--2017}{Software Engineer}{Working on Thesis}{\href{https://www.codilime.com/}{\underline{CodiLime}}}{Warsaw, Poland}{\newline}}

\textbf{\cventry{summer 2016}{Software Engineer}{Internship}{\href{https://intel.com}{\underline{Intel}}}{Gdańsk, Poland}{}}
\cvitem{description}{Frontend for internal software testing tool}
\cvitem{technologies}{Angular, Flask, Python, Microsoft SQL Server}

\section{\textsc{Education}}
\textbf{\cventry{2018--2019}{Computer Science}{}{University of Warsaw}{\textit{Master's degree}}{}}
\cvitem{thesis}{\textit{\href{https://play.google.com/store/apps/details?id=io.storage.cloudbrowser}{\underline{Cloud Browser}}}}
\cvitem{description}{File manager supporting multiple cloud storage providers with built-in video and music player}
\cvitem{technologies}{Android SDK, Kotlin, C++\newline}

\textbf{\cventry{2014--2018}{Computer Science}{}{University of Warsaw}{\textit{Bachelor's degree}}{}}
\cvitem{thesis}{\textit{Scalable Graph Algorithms}}{}
\cvitem{description}{Scalable versions of graph algorithms implementation and performance comparison with single-core equivalents}
\cvitem{technologies}{C++, \href{https://developer.nvidia.com/thrust}{\underline{Thrust}}, CUDA, googletest}

\section{\textsc{Skills}}
\cvitem{languages}{Polish, English}
\cvitem{web}{HTML, CSS, JavaScript, React}
\cvitem{programming}{Python, Kotlin, Java, C, C++, Haskell}
\cvitem{frameworks}{Django, Android SDK}
\cvitem{databases}{PostgreSQL, SQLite, MySQL}
\cvitem{DevOps}{Linux, Bash, Zsh, Ansible}
\cvitem{build systems}{GNU make, Gradle}
\cvitem{other}{\LaTeX\xspace, git}

\newpage

\section{\textsc{Personal/University Projects}}
\cvitem{name}{Sokoban}
\cvitem{description}{Logic game made for an assignment at Android University Course.
  Implemented in 2016 in Java, later refactored in Kotlin}
\cvitem{links}{\href{https://play.google.com/store/apps/details?id=com.starsep.sokoban.release}{\underline{Google Play}}, \href{https://github.com/starsep/Sokoban}{\underline{GitHub}}, \href{https://en.wikipedia.org/wiki/Sokoban}{\underline{Wikipedia}}\newline}

\cvitem{name}{2048}
\cvitem{description}{Simple 2048 clone implemented to test \href{https://unity.com/}{\underline{Unity}} framework}
\cvitem{links}{\href{https://play.google.com/store/apps/details?id=com.starsep.game2048}{\underline{Google Play}}, \href{https://github.com/starsep/2048}{\underline{GitHub}}\newline}}

\cvitem{name}{Latte compiler}
\cvitem{description}{Compiler in Haskell for Java-like language Latte to x86-64 NASM assembly}
\cvitem{links}{\href{https://github.com/starsep/latte}{\underline{GitHub}}\newline}}

\cvitem{name}{Mustache In The Middle}
\cvitem{description}{HTTP Proxy which adds mustaches to photos. Implemented during \href{https://hackyeah.pl/}{\underline{HackYeah}} hackathon}
\cvitem{links}{\href{https://github.com/starsep/mustache}{\underline{GitHub}}, \href{https://en.wikipedia.org/wiki/Man-in-the-middle_attack}{\underline{Man-in-the-middle attack Wikipedia article}}}


\end{document}
