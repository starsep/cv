%% start of file `template.tex'.
%% Copyright 2006-2012 Xavier Danaux (xdanaux@gmail.com).
%
% This work may be distributed and/or modified under the
% conditions of the LaTeX Project Public License version 1.3c,
% available at http://www.latex-project.org/lppl/.

\documentclass[11pt,a4paper,sans]{moderncv}        % possible options include font size ('10pt', '11pt' and '12pt'), paper size ('a4paper', 'letterpaper', 'a5paper', 'legalpaper', 'executivepaper' and 'landscape') and font family ('sans' and 'roman')

\usepackage{verbatim}
\usepackage{polski}
\usepackage[utf8]{inputenc}
\usepackage[firstyear=2008,lastyear=2015]{moderntimeline}
\usepackage{graphicx}
\usepackage{epstopdf} %%package to overcome problem with eps in pdf files
\usepackage{float}
\restylefloat{figure}

\newcommand*{\cvtripleitem}[7][.25em]{%
 \cvitem[#1]{#2}{%
   \begin{minipage}[t]{\tripleitemmaincolumnwidth}#3\end{minipage}%
   \hfill
   \begin{minipage}[t]{\hintscolumnwidth}\raggedleft\hintstyle{#4}\end{minipage}%
   \hspace*{\separatorcolumnwidth}%
   \begin{minipage}[t]{\tripleitemmaincolumnwidth}#5\end{minipage}%
   \hfill
   \begin{minipage}[t]{\hintscolumnwidth}\raggedleft\hintstyle{#6}\end{minipage}%
   \hspace*{\separatorcolumnwidth}%
   \begin{minipage}[t]{\tripleitemmaincolumnwidth}#7\end{minipage}%
   }%
}

\newcommand{\doublecventrylogo}[9][.25em]{% 
\cvdoubleitem[#1]{#2}
  {\hfill \raisebox{-0.5\height}{\includegraphics[width=#4]{#3}}}
  {#5}
  {\hfill \raisebox{-0.5\height}{\includegraphics[width=#7]{#6}}}
}

    
% moderncv themes
\moderncvstyle{classic}                             % style options are 'casual' (default), 'classic', 'oldstyle' and 'banking'
\moderncvcolor{red}                               % color options 'blue' (default), 'orange', 'green', 'red', 'purple', 'grey' and 'black'
%\renewcommand{\familydefault}{\sfdefault}         % to set the default font; use '\sfdefault' for the default sans serif font, '\rmdefault' for the default roman one, or any tex font name
%\nopagenumbers{}                                  % uncomment to suppress automatic page numbering for CVs longer than one page

% adjust the page margins
\usepackage[scale=0.75]{geometry}
%\setlength{\hintscolumnwidth}{3cm}                % if you want to change the width of the column with the dates
%\setlength{\makecvtitlenamewidth}{10cm}           % for the 'classic' style, if you want to force the width allocated to your name and avoid line breaks. be careful though, the length is normally calculated to avoid any overlap with your personal info; use this at your own typographical risks...

% personal data
\firstname{Filip}
\familyname{Czaplicki}
\title{Resumé title}                               % optional, remove / comment the line if not wanted
\address{Klarnecistów 27}{02-875}{Warsaw Poland}% optional, remove / comment the line if not wanted; the "postcode city" and and "country" arguments can be omitted or provided empty
\mobile{+48~600~131~257}                          % optional, remove / comment the line if not wanted
%\phone{+48~()~678~901}                           % optional, remove / comment the line if not wanted
%\fax{+3~(456)~789~012}                             % optional, remove / comment the line if not wanted
\email{filipczaplicki@gmail.com}                               % optional, remove / comment the line if not wanted
\homepage{starsep.com}                         % optional, remove / comment the line if not wanted
%\extrainfo{additional information}                 % optional, remove / comment the line if not wanted
\photo[64pt][0.4pt]{picture}                       % optional, remove / comment the line if not wanted; '64pt' is the height the picture must be resized to, 0.4pt is the thickness of the frame around it (put it to 0pt for no frame) and 'picture' is the name of the picture file
%\quote{Some quote}                                 % optional, remove / comment the line if not wanted

% to show numerical labels in the bibliography (default is to show no labels); only useful if you make citations in your resume
%\makeatletter
%\renewcommand*{\bibliographyitemlabel}{\@biblabel{\arabic{enumiv}}}
%\makeatother

% bibliography with mutiple entries
%\usepackage{multibib}
%\newcites{book,misc}{{Books},{Others}}
%----------------------------------------------------------------------------------
%            content
%----------------------------------------------------------------------------------

\begin{document}

\makecvtitle

\section{Education}
%\cventry{year--year}{Degree}{Institution}{City}{\textit{Grade}}{Description}  % arguments 3 to 6 can be left empty
%\cventry{year--year}{Degree}{Institution}{City}{\textit{Grade}}{Description}
\tlcventry{2014}{0}{Computer Science}{University of Warsaw}{}{}{}
\tlcventry{2011}{2014}{High School}{Polish Navy High School}{Gdynia}{Math and CS Class}{}
\tlcventry{2008}{2011}{Junior High School}{I Gimnazjum}{Tczew}{Advanced English Class}{}
%\section{Master thesis}
%\cvitem{title}{\emph{Title}}
%\cvitem{supervisors}{Supervisors}
%\cvitem{description}{Short thesis abstract}

\begin{comment}
\section{Experience}
\subsection{Vocational}
\cventry{year--year}{Job title}{Employer}{City}{}{General description no longer than 1--2 lines.\newline{}%
Detailed achievements:%
\begin{itemize}%
\item Achievement 1;
\item Achievement 2, with sub-achievements:
  \begin{itemize}%
  \item Sub-achievement (a);
  \item Sub-achievement (b), with sub-sub-achievements (don't do this!);
    \begin{itemize}
    \item Sub-sub-achievement i;
    \item Sub-sub-achievement ii;
    \item Sub-sub-achievement iii;
    \end{itemize}
  \item Sub-achievement (c);
  \end{itemize}
\item Achievement 3.
\end{itemize}}
\cventry{year--year}{Job title}{Employer}{City}{}{Description line 1\newline{}Description line 2}
\subsection{Miscellaneous}
\cventry{year--year}{Job title}{Employer}{City}{}{Description}
\end{comment}

\section{Languages}
\cvitemwithcomment{Polish}{Native}{}
\cvitemwithcomment{English}{Upper Intermediate}{}
%\cvitemwithcomment{Language 3}{Skill level}{Comment}

\section{Computer skills}
Technologies I feel confident with: \\
\doublecventrylogo{Java}{images/java.eps}{5mm}{C++}{images/cpp.eps}{12mm}{}{}
\doublecventrylogo{Java}{images/java.eps}{5mm}{C++}{images/cpp.eps}{12mm}{}{}
Technologies I can do something with: \\
\doublecventrylogo{Java}{images/java.eps}{5mm}{C++}{images/cpp.eps}{12mm}{}{}
\doublecventrylogo{Java}{images/java.eps}{5mm}{C++}{images/cpp.eps}{12mm}{}{}
Technologies I know something about: \\
\doublecventrylogo{Java}{images/java.eps}{5mm}{C++}{images/cpp.eps}{12mm}{}{}
\doublecventrylogo{Java}{images/java.eps}{5mm}{C++}{images/cpp.eps}{12mm}{}{}
\section{Interests}
\cvitem{Contract bridge}{Description}
\cvitem{Board games}{Description}
\cvitem{Music}{Description}

\section{Extra 1}
\cvlistitem{Item 1}
\cvlistitem{Item 2}
\cvlistitem{Item 3. This item is particularly long and therefore normally spans over several lines. Did you notice the indentation when the line wraps?}

\end{document}

